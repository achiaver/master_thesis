% ----------------------------------------------------------
% ELEMENTOS PRÉ-TEXTUAIS
% ----------------------------------------------------------
\frontmatter

\capa
\folhaderosto

% ----------------------------------------------------------
% FICHA CATALOGRÁFICA

% A biblioteca deverá providenciar a ficha catalográfica. Salve a ficha no formato PDF.
% Use o nome do arquivo PDF como argumento do comando. Exemplo: ficha catalográfica
% no arquivo 'ficha.pdf'

% \fichacatalografica{ficha.pdf}

% Enquanto não possuir a ficha catalográfica, use o comando sem argumentos...
\fichacatalografica{}

% ----------------------------------------------------------
% FOLHA DE APROVAÇÃO
% membros da banca: máximo 6

\begin{folhadeaprovacao}%
    \assinatura{Dra. Maria Clicia Stelling de Castro}{IME/CComp – UERJ}
    \assinatura{Dr. Diego Nunes Brandão}{EIC/PPCIC – CEFET/RJ}
    % \assinatura{~ }{~ } % para se ter linha em branco
\end{folhadeaprovacao}

% ----------------------------------------------------------
% DEDICATÓRIA
% \pretextualchapter{Dedicatória}
% \vfill
% Texto da dedicatória

% ----------------------------------------------------------
% AGRADECIMENTOS
\pretextualchapter{Acknowledgements}

\noindent
\begin{minipage}{\textwidth}

First of all, I'd like to thank all of those who have been waiting for the end of this tough course.
The supports came on different matters from each particular effort someone can make.
Fortunately, patience was key for all of us to conduct learning over each other mistakes.
\bigskip

I would like to thank both advisors, Igor Machado Coelho and Vitor Nazário Coelho, for the time given to develop the skills for self-criticism, writing, and workarounds out of the barriers faced on the way.
Special thanks also go to their life partners, Cristiane Tavares and Thays Aparecida de Oliveira, for random conversations on unusual meetings.
\bigskip

Moreover, this thesis would have been impossible without the contributions of Luis (Nando) Ochoa, Luiz Alberto Fortunato, Sérgio Mafra, and Nivalde José de Castro.
With the sharing of their scarce time to appoint me valuable reports and to allow discussions about the subject at the beginning of the present work, they had contributed to reaching further horizons.
Similar appreciation goes for Alexandre Sztajnberg, Luciano Porto Barreto, Maria Clicia Stelling de Castro, and Diego Nunes Brandão, who have integrated both examination committees, the mid-term and the final one, and have given important guidance to the research structure and content presentation.
\bigskip

Also, I am particularly grateful for the support given by professors and assistants from the Computational Science department that were willing to help me during the whole pathway.
Likewise, I would like to express my very great appreciation to colleagues for sharing their knowledge far beyond what classrooms may offer.
\bigskip

Above all, I wish to acknowledge the help provided by my closest relatives and friends.
At every uncertain day, they kept strong the support to reach the desired outcome.
The support that is only given by few but that surpasses in commitment and in confidence the difficulties imposed by others.

\end{minipage}
\vfill

% ----------------------------------------------------------
% EPÍGRAFE (opcional)
\pretextualchapter{}
\vfill

% \begin{flushright}
% \begin{quote}
\noindent
% As novas tecnologias nunca vêm sozinhas.
% É um pacote: mudanças tecnológicas, seguidas de mudanças sociais, políticas e culturais.
New technologies never come alone.
It's a package: technological changes, followed by social, political and cultural changes.
% \end{quote}

\hfill\textit{Alvin Toffler (The Third Wave)}
% \end{flushright}

% ----------------------------------------------------------
% RESUMO
\pretextualchapter{Resumo}
\referencia

Sabe-se que o aumento do consumo de eletricidade é consequência direta da melhoria da qualidade de vida e bem-estar da população.
Desde grandes usinas centralizadas à Geração Distribuída (DG) de eletricidade, procura-se suprir essa demanda de energia por meio de alternativas mais sustentáveis e socialmente responsáveis.
Apesar dos desafios técnicos e comerciais, o avanço da Tecnologia da Informação e Comunicações (TIC) está trazendo novas abordagens para o setor elétrico.
Por exemplo, a integração de blockchain com sistemas de energia pode cobrir algumas das dificuldades identificadas até o momento.
O presente trabalho tenta mostrar como um aplicativo distribuído (Dapp) pode apoiar o gerenciamento de energia de micro/mini redes de forma a permitir a sua comercialização (Energia Transativa -- TE) dentro da categoria brasileira de geração compartilhada.

\imprimirchaves

% ----------------------------------------------------------
% ABSTRACT
\pretextualchapter{Abstract}
\reference

% quasi-POC that outlines

It is well known that increasing electricity consumption is a direct consequence of betterment of the population quality of life and well-being.
From centralized big power plants to \acrfull{gd} of electricity, there is a look for supply this power demand employing more sustainable and socially responsible alternatives.
Despite the technical and business challenges to do so, the advance of \acrfull{ict} has bringing new approaches to the electric sector.
For instance, the integration of blockchain with power systems can cover some of the hindrances identified so far.
The present work attempts to show how a \acrfull{dapp} can support the power management of micro/mini-grids towards a \acrfull{te} inside the Brazilian market design of shared generation.

\printkeys

% ----------------------------------------------------------
% LISTA DE ILUSTRAÇÕES E DE TABELAS
\listadefiguras
\listadetabelas

% ----------------------------------------------------------
% OUTRAS LISTAS (opcionais)
% \listadealgoritmos

% ----------------------------------------------------------
% LISTA DE ABREVIATURAS E SIGLAS
\printglossaries
% \printnoidxglossaries

% \acrlong{ }
% \acrshort{ }
% \acrfull{ }

% ----------------------------------------------------------
% SUMÁRIO
\sumario
