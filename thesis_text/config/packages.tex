\usepackage[brazil,american]{babel}	% adequação para o português Brasil
\hyphenation{block-chain de-ci-sion-mak-er frame-works ti-me-fra-me}

\usepackage[T1]{fontenc}            % to get right quotes on listing
\usepackage[utf8x]{inputenc}		% etermina a codificação utilizada
								    % utf8 = (conversão automática dos acentos)
								    % utf8x = extended unicode types

\usepackage{pgffor} 				% required for \foreach (loop statement in LaTeX)
\usepackage{chngpage}			    % allows for temporary adjustment of side margins
\usepackage{titlecaps}              % required for \titlecap

\usepackage{makeidx}			    % Cria o índice
\usepackage{hyperref}			    % Influencia na formação do índice
\usepackage{indentfirst}	        % Indenta o primeiro parágrafo de cada seção.

\usepackage{graphicx}			    % Inclusão de gráficos
%\graphicspath{%
%{images/}                       % Usar o nome/caminho da pasta onde estao as images
%}

\usepackage[table]{xcolor}		    % Permite criar variáveis com cores,
                                    %   a opção habilita cores em tabelas
% \let\oldtabular\tabular             % tabela cor-sim/cor-não
% \let\endoldtabular\endtabular
% \renewenvironment{tabular}{\rowcolors{2}{white}{lightgray}\oldtabular}{\endoldtabular}

\usepackage[inline]{enumitem}	    % Diversos tipos de listas
\usepackage{multirow}               % Permite unir linhas e colunas de uma tabela
\usepackage{amsmath}			    % Pacote matemático
\usepackage{bm}                     % bold text on math mode
\makeatletter
\@fleqnfalse
\@mathmargin\@centering
\makeatother

\usepackage[subrefformat=parens, labelformat=parens]{subfig}
\captionsetup[subfigure]{justification=centering}
\newcommand{\subfigureautorefname}{\figureautorefname}
\newcommand{\lstnumberautorefname}{\figureautorefname}

\usepackage{tikz}				    % Awesome drawings
\usetikzlibrary{calc}               % to get node width and height
\usetikzlibrary{positioning}        % to use the relative positioning commands
\usetikzlibrary{arrows,arrows.meta}
\usetikzlibrary{mindmap}
\usetikzlibrary{shapes.geometric}   % to use diamond shape
\usetikzlibrary{backgrounds}        % for frames
\usetikzlibrary{shadows}            % for shadows
\usetikzlibrary{decorations.pathmorphing,decorations.pathreplacing}
\usetikzlibrary{fit}
\usetikzlibrary{petri}
\usetikzlibrary{matrix}
\usetikzlibrary{chains}

\usepackage{listings}               % For codes environment
% \lstloadlanguages[Sharp]{C}
% \lstset{defaultdialect=[Sharp]C}

\usepackage[acronym, nonumberlist, nopostdot]{glossaries}
\newacronym{aneel}{ANEEL}
% {Agência Nacional de Energia Elétrica}
{Brazilian Electricity Regulatory Agency}

\newacronym{ccee}{CCEE}
% {Câmara de Comercialização de Energia Elétrica}
{Electric Power Trading Chamber}

\newacronym{epe}{EPE}
% {Empresa de Pesquisa Energética}
{Energy Research Company}

\newacronym{mme}{MME}
% {Ministério de Minas e Energia}
{Brazilian Ministry of Mines and Energy}

\newacronym{sin}{SIN}
% {Sistema Interligado Nacional}
{National Power Grid System}

\newacronym{cmse}{CMSE}
% {Comitê de Monitoramento do Setor Elétrico}
{Power Sector Monitoring Committee}

\newacronym{ons}{ONS}
% {Operador Nacional do Sistema Elétrico}
{National Power System Operator}

\newacronym{acr}{ACR}
% {Ambiente de Contratação Regulada}
{Regulated Contracting Environment}

\newacronym{acl}{ACL}
% {Ambiente de Contratação Livre}
{Free Contracting Environment}

\newacronym{seb}{SEB}
% {Setor Elétrico Brasileiro}
{Brazilian Electrical Sector}

\newacronym{iaas}{IaaS}
% {Infraestrutura como um serviço}
{Infrastructure as a Service}

\newacronym{paas}{PaaS}
% {Plataforma como um serviço}
{Platform as a Service}

\newacronym{saas}{SaaS}
% {Software como um serviço}
{Software as a Service}

\newacronym{baas}{BaaS}
% {Blockchain como um serviço}
{Blockchain as a Service}

\newacronym{reseb}{Projeto RE-SEB}
% {Projeto de Reestruturação do Setor Elétrico Brasileiro}
{Restructuring of the Brazilian Electricity Sector Project}

\newacronym{der}{DER}
% {Recursos Energéticos Distribuídos}
{Distributed Energy Resources}

\newacronym{gd}
% {GD}{Geração Distribuída}
{DG}{Distributed Generation}

\newacronym[plural=DLTs]{dlt}{DLT}% ,firstplural=Distributed Ledger Technologies (DLTs)
% {Tecnologia de Registros Distribuída}
{Distributed Ledger Technology}

\newacronym[plural=Dapps,firstplural=distributed applications (Dapps)]{dapp}{Dapp}
% {aplicação distribuída}
{distributed application}

\newacronym{Dapp}{Dapp}
% {Aplicação Distribuída}
{Decentralized Application}

\newacronym{p2p}{P2P}
% {}
{peer-to-peer}

\newacronym{mcp}{MCP}
% {Mercado de Curto Prazo}
{}

\newacronym{pis}{PIS}
% {Programas de Integração Social e de Formação do Patrimônio do Servidor Público}
{}

\newacronym{cofins}{COFINS}
% {Contribuição para Financiamento da Seguridade Social}
{}

\newacronym{icms}{ICMS}
% {Imposto sobre Circulação de Mercadorias e Serviços}
{}

\newacronym{ee}{EE}
% {Energia Elétrica}
{Electric Energy}

\newacronym{fv}{FV}
% {Energia Elétrica Fotovoltaica}
{}

\newacronym{pv}{PV}
% {}
{Photovoltaic}

\newacronym{proinfa}{PROINFA}
% {Programa de Incentivo às Fontes Alternativas de Energia Elétrica}
{}

\newacronym{its}{ITS Rio}
% {Instituto de Tecnologia e Sociedade do Rio}
{}

\newacronym{b2b}{B2B}
% {}
{business-to-business}

\newacronym{c2c}{C2C}
% {}
{consumer-to-consumer}

\newacronym{b2c}{B2C}
% {}
{business-to-consumer}

\newacronym{iot}{IoT}
% {}
{Internet Of Things}

\newacronym{tic}{TIC}
% {Tecnologias da Informação e Comunicação}
{}

\newacronym[plural=ICTs, firstplural=Information and Communications Technologies (ICTs)]{ict}{ICT}
% {Tecnologia da Informação e Comunicação}
{Information and Communications Technology}

\newacronym{pow}{PoW}
% {}
{Proof of Work}

\newacronym{pos}{PoS}
% {}
{Proof of Stake}

\newacronym{ico}{ICO}
% {}
{Initial Coin Offering}

\newacronym{poet}{PoET}
% {}
{Proof of Elapsed Time}

\newacronym{pob}{PoB}
% {}
{Proof of Burn}

\newacronym{por}{PoR}
% {}
{Proof of Retrievability}

\newacronym{poa}{PoA}
% {}
{Proof of Authority}

\newacronym{tee}{TEE}
% {}
{Trusted Execution Environment}

\newacronym{pbft}{PBFT}
% {}
{Practical Byzantine Fault Tolerance}

\newacronym{sbft}{SBFT}
% {}
{Simplified Byzantine Fault Tolerant}

\newacronym{dbft}{DBFT}
% {}
{Delegated Byzantine Fault Tolerance}

\newacronym{mtemsm}{MTEMsm}
% {}
{Microgrid Transactive Energy Management Smart Contract}

\newacronym{pp}{\emph{PP}}
% {}
{Power Plant}

\newacronym{tg}{\emph{TG}}
% {}
{Token Generator}

\newacronym{rum}{\emph{Rum}}
% {}
{Referendum}

\newacronym{mem}{\emph{Member}}
% {}
{Member}

\newacronym[plural=APIs, firstplural=Application Programming Interfaces (APIs)]{api}{API}
% {Interface de Programação de Aplicações}
{Application Programming Interface}

\newacronym{uml}{UML}
% {Linguagem de Modelagem Unificada}
{Unified Modeling Language}

\newacronym{te}{TE}
% {Energia Transativa}
{Transactive Energy}
				    % Lista de acrônimos
\makeglossaries

\usepackage[alf]{abntex2cite}       % Pacotes de citações úteis para o formato ABNT
                                    %   (gera erro no BibTeX)

% ----------------------------------------------------------
% Pacote auxiliar para as normas da UERJ
% --- (precisa ficar aqui) ---
\usepackage[frame=no,algline=yes,font=default]{config/UERJ/repUERJformat}
\usepackage{config/UERJ/repUERJpseudocode}
% ----------------------------------------------------------
