% Altera o título de algumas seções e de algumas tags
\addto\captionsamerican{%
    \renewcommand{\acronymname}{List of abbreviations and acronyms}%
    \renewcommand*{\glossarysection}[2][]{\pretextualchapter{#2}}%
}

\addto\captionsbrazil{%
    \renewcommand{\acronymname}{\abrevname}%
}

% Colors
\definecolor{lightgray}{gray}{0.9}

% Table configs
% \renewcommand{\arraystretch}{1.2}

% Algorithm environment
\lstdefinestyle{mystyle}{
    language=[Sharp]C,
    xleftmargin=7mm,
    basicstyle=\scriptsize,
    breakatwhitespace=false,
    breaklines=true,
    postbreak=\mbox{\textcolor{gray}{$\hookrightarrow$}\space},
    captionpos=t,
    % abovecaptionskip=?pt,
    % belowcaptionskip=?pt,
    keepspaces=true,
    numbers=left,
    numbersep=5pt,
    numberstyle={\color{gray}\sffamily\tiny},
    showspaces=false,
    showstringspaces=false,
    showtabs=false,
    tabsize=2,
    framexleftmargin=6mm,
    frame=single
}
\lstset{style=mystyle}
\renewcommand{\lstlistingname}{Code}



% Acrônimo para se usar dentro de lista
\newcommand{\glsitem}[1]{\glsreset{#1}\gls{#1}}

% Lista customizada
% (inline list with bold label with items separated by semicolon)
\newenvironment{descriptive}{\begin{itemize*}[font=\bfseries, before=\unskip{: }, itemjoin={{; }}, itemjoin*={{; and }}, afterlabel={{ -- }}]}{\end{itemize*}}

% Configuration of list inside table.
\newcommand{\obs}[5][Base58]{%
    \begin{description}[itemsep=0pt, parsep=0pt, leftmargin=0pt]
        \item[ID] {\ttfamily #2}
        \item[format] #1
        \item[belongs to] #3
        \item[created at] #4
        \item[note] #5
    \end{description}%
}

% Coloca a primeira letra da referência em maiúscula
%   acho q não está funcionando direito
%   (pelo menos não faz diferença para o 'autoref')
\let\orgautoref\autoref
\providecommand{\Autoref}[1]{%
    \def\appendixautorefname{Appendix}%
    \def\sectionautorefname{Section}%
    \def\chapterautorefname{Chapter}%
    % \def\postextualchapterautorefname{Appendix}%
    \orgautoref{#1}%
}

% Marcação de direitos autorais
\def \copyright{The author, 2020.}
% \textcopyright\space \UERJautornome\space\UERJautorsobrenome,\space2019.

% Marcação de licença
\def \license{\href{https://creativecommons.org/licenses/by/4.0/}{CC BY 4.0}}

% Relacionamento entre a descrição das operações com as imagens dos fluxogramas
\def \good#1{\textbf{\MakeUppercase{#1}}\\%
    \small%
    \ifthenelse{\equal{\detokenize{#1}}{\detokenize{summary}}}%
    {\textit{(Figures \ref{fig:#11} to \ref{fig:#12})}}%
    {% else 1
    \ifthenelse{\equal{\detokenize{#1}}{\detokenize{change}}}%
    {\textit{(Figures \ref{fig:#11} to \ref{fig:#14})}}%
    {% else 2
    \ifthenelse{\equal{\detokenize{#1}}{\detokenize{power-up}}}%
    {\textit{(Figures \ref{fig:#11} to \ref{fig:#13})}}%
    {% else 3
    \textit{(\autoref{fig:#1})}%
    }}}% else's 1,2,3
    }

% Auto numbering of rows.
\newcounter{rowno}
\setcounter{rowno}{0}

% Base58 address formatting
\newcommand{\ID}[1]{\texttt{#1}}

% ----------------------------------------------------------
% Configuração padrão para os fluxogramas
\newcommand{\configTikzFlowcharts}{
    \tikzset{%
        >={Latex[width=2mm,length=2mm]},
        node distance=1.5cm,
        % inner frame sep=0.5cm, show background rectangle --> to be implemented
        % Specifications for style of nodes:
        base/.style = {fill=white, draw=black,
            %yshift=-1cm,
            %minimum width=2cm, minimum height=1cm,
            text width=2.5cm, text centered, font=\sffamily},
        onchain/.style = {base, rectangle, rounded corners},
        offchain/.style = {onchain, dashed},
        terminal/.style = {base, text width=3cm, ellipse},
        i-o/.style = {base, trapezium, trapezium left angle=60, trapezium right angle=120},
        i2o/.style = {i-o, text width=5cm},
        decision/.style = {base, diamond},
        erase/.style = {fill=white, shape=circle, minimum size=1.8*\radius, inner sep=0pt},
    }
}

\tikzstyle{dash} = [anchor=west, draw, rounded corners=0.1cm, dashed, font=\scriptsize]
\tikzstyle{straight} = [anchor=west, draw, rounded corners=0.1cm, thick, font=\scriptsize]
\tikzstyle{remaining}=[font=\scriptsize]

% Figures configurations
\tikzstyle{print} = [drop shadow={gray!50, shadow xshift=3pt, shadow yshift=-3pt, rounded corners }, rounded corners, draw=gray, fill=white, text width=.95\linewidth]

\tikzstyle{borda} = [framed, background rectangle/.style={fill=white}, inner frame sep=3pt, outer frame sep=0]

\tikzstyle{pick} = [thick, rounded corners, red!80!black, font=\footnotesize]

\tikzstyle{r} = [>=stealth, ->, thick, red!80!black]

% UML configurations
\newcommand{\umlblock}[4][2.5cm]{%
    \frame{%
        \begin{minipage}[t][#1]{\ifdim#4=0cm 4cm \else #4 \fi}
            {\centering
            \vspace{4pt}
            \makebox[1.1\width][c]{#2}\vspace{3pt}
            \hrule}
            \vspace{3pt}
            \begin{tabular}{l}
                #3
            \end{tabular}
        \end{minipage}%
    }
}

\newcounter{id}
\newcommand{\id}{\refstepcounter{id}\large\textsf{\Alph{id}}}

\newcommand{\hrel}[5][16pt]{\draw[-] (#2) node[L, inner sep=#1] at (#2.east) {#3} -- (#4) node[R, inner sep=#1] at (#4.west) {#5};}
\newcommand{\vrel}[5][8pt]{\draw[-] (#2) node[align=right, anchor=east, yshift=-#1, xshift=18pt] at (#2.south) {#3} -- (#4) node[align=right, anchor=east, yshift=#1, xshift=18pt] at (#4.north) {#5};}

\tikzset{detach/.style={draw=black!80!white, rounded corners, dashed, thick}}
\tikzset{ID/.style={anchor=north east}}
\tikzset{L/.style={align=left, anchor=west}}
\tikzset{R/.style={align=right, anchor=east}}

% VALUABLE NOTES
% \draw[->]     (d) -- ++(-4cm,0) |- (end) node[pos=0.25] {No};
% [node distance=1.5cm, align=center]
