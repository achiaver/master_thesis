% Smart Grid + VPP + Energy Internet + Transactive Grid + SEN (1 parágrafo para cada assunto + 1 parágrafo conectando eles)

\section*{Smart Grid}


\section*{Virtual Power Plant}


\section*{Energy Internet}


\section*{Transactive Energy}
% Transactive energy (TE) can be defined as “a system of economic and control mechanisms that allows the dynamic balance of supply and demand across the entire electrical infrastructure using value as a key operational parameter.”  This definition, which is the one currently used by NIST, was originally proposed by the U.S. Department of Energy’s Gridwise Architecture Council in its Transactive Energy Framework (link is external).
% %https://www.nist.gov/engineering-laboratory/smart-grid/transactive-energy-overview

% The term "transactive energy" is used here to refer to techniques for managing the generation, consumption or flow of electric power within an electric power system through the use of economic or market based constructs while considering grid reliability constraints. The term "transactive" comes from considering that decisions are made based on a value. These decisions may be analogous to or literally economic transactions. An example of an application of a transactive energy technique is the double auction market used to control responsive demand side assets in the GridWise 
% %https://www.gridwiseac.org/about/transactive_energy.aspx

\section*{Smart Energy Network}


% GENERAL CONSEQUENCES AND GAPS