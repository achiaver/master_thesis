% =============================================
% Stakeholder Requirements Specification (StRS)
% =============================================

	- describes the organization's motivation for why the system is being developed
    - defines processes and policies/rules under which the system is used
    - documents the top level requirements from the stakeholder perspective including needs of users/operators/maintainers as derived from the context of use
    
In a business environment:
	- describes how the organization is pursuing new business or changing the current business in order to fit a new business environment
    - how to utilize the system as a means to contribute to the business
    - The description at the organization level includes:
    	- the organizational environment, goals and objectives,
        - the business model,
        - the information environment
    - The description at the business operation level includes:
    	- the business operation model,
        - the business operation modes,
        - the business operational quality,
        - the organizational formation,
        - the concept of the proposed system.
        
\emph{The information items of the StRS should be specified by the stakeholders}
	- necessário identificá-los corretamente
    - não temos a opinião de todos (a proposta é generalista!)

Typical types of stakeholder requirements included in the StRS are:
	- organizational requirements,
    - business requirements,
    - user requirements.

NOTE 1 ISO/IEC/IEEE 15289 provides guidance to include business, organizational, and user (stakeholder)
requirements in the system requirements specification. This International Standard includes these requirements in the
StRS since the contents should be specified from the stakeholders' perspective. They may be succeeded in the SyRS by addressing technical concerns.

	"Throughout this document, “shall” is used	 to	 express	 a	 provision	 that	 is	 normative,	 “should”	 to	 express	 a	recommendation	 among	 other	 possibilities,	 and	 “may”	 to	 indicate	 a	 course	 of action permissible	 within	 the	limits	of	this	document."
    
    "The	verb	 “include”	used	in	 this	document	indicates	 that	either	(1)	 the	information	is	present	or	 (2)	 a	reference to	the	information	is	given."
    
    "The	information	items	 (documents)	are	produced	and	communicated	for	human	use	and	contain	formal	elements	(such	as	purpose,	scope,	and	summary), intended	to	make	them	usable	by	their	intended audience."
    
    "The	 Information	 Management	 process	 should	support	 the	 needs	 of	 a	 project	 and	 the	 related	 product	 or	 service.	 It	 should	 include	 procedures	 for	 preparing,	collecting,	identifying,	classifying,	distributing,	storing,	updating,	archiving,	and	retrieving	information." [6.4 anx.A]
    
    "7 Generic types of information items"
    
    COMEÇAR PELO ANEXO A!
    
    https://sci-hub.tw/https://ieeexplore.ieee.org/document/7942151/
    

NOTE 3 The stakeholder requirements and business requirements are distinguished in The Guide to the Business
Analysis Body of Knowledge (BABOK) as follows: Business Requirements are high-level statements of the goal, objectives,
or needs of the enterprise. They describe why a project is initiated, what the project will achieve, and which metrics will be used to measure the project's success. Stakeholder Requirements are statements of the needs of a particular stakeholder
or class of stakeholders. They describe the needs that a given stakeholder has and how that stakeholder will interact with
a solution. Stakeholder Requirements serves as a bridge between Business Requirements and the various classes of
solution requirements.

% ----------------------------------------------------------
Example StRS Outline

1. Introduction
  1.1 Business purpose
  1.2 Business scope
  1.3 Business overview
  1.4 Definitions
  1.5 Stakeholders
2. References
3. Business management requirements
  3.1 Business environment
  3.2 Goal and objective
  3.3 Business model
  3.4 Information environment
4. Business operational requirements
  4.1 Business processes
  4.2 Business operational policies and rules
  4.3 Business operational constraints
  4.4 Business operational modes
  4.5 Business operational quality
  4.6 Business structure
5. User requirements
6. Concept of proposed system
  6.1 Operational concept
  6.2 Operational scenario
7 Project Constraints
8. Appendix
  8.1 Acronyms and abbreviations
% ----------------------------------------------------------


Na geração distribuída temos os seguintes stakeholders:
	- end users:
        - autoconsumo local
        	- prosumer
        - autoconsumo remoto
        	- prosumer
       	- múltiplas unidades consumidoras (system acquirer)
        	- prosumers
            - consumers
        - geração compartilhada (system acquirer)
        	- prosumers
            - consumers
    - system supplier:
    	- DISSERTAÇÃO
    - regulatory bodies:
    	- ANEEL
        - distribuidora
        - cooperativa ou condomínio (estatuto e regulamentação)



% Note the use of \path instead of \node at ... below.
% \path (naveq.140)+(-\blockdist,0) node (gyros) [sensor] {System Element};
% \path (naveq.-150)+(-\blockdist,0) node (accel) [sensor] {System Element};

% Unfortunately we cant use the convenient \path (fromnode) -- (tonode) 
% syntax here. This is because TikZ draws the path from the node centers
% and clip the path at the node boundaries. We want horizontal lines, but
% the sensor and naveq blocks aren't aligned horizontally. Instead we use
% the line intersection syntax |- to calculate the correct coordinate
% \path [draw, ->] (gyros) -- node [above] {$\vc{\omega}_{ib}^b$} (naveq.west |- gyros);
% We could simply have written (gyros) .. (naveq.140). However, it's
% best to avoid hard coding coordinates
% \path [draw, ->] (accel) -- node [above] {$\vc{f}^b$} (naveq.west |- accel);

% \draw [->] (naveq.50) -- node [ann] {Velocity } + (\edgedist,0) node[right] {$\vc{v}^l$};
% \draw [->] (naveq.20) -- node [ann] {Attitude} + (\edgedist,0) node[right] { $\mx{R}_l^b$};
% \draw [->] (naveq.-25) -- node [ann] {Horisontal position} + (\edgedist,0) node [right] {$\mx{R}_e^l$};
% \draw [->] (naveq.-50) -- node [ann] {Depth} + (\edgedist,0) node[right] {$z$};