%---------------------------------------------------------
% muito específico do mercado, mas fora da tese
%---------------------------------------------------------

% POLÍTICAS DE PREÇO

% Another important feature in this scenario, aiming at price modesty, is \acrshort{ee} auctions, which act as a power purchase tool for distributors in \acrshort{acr}. The auctions are performed by \acrshort{ccee}, by \acrshort{aneel} delegation, and use the lowest tariff criterion, aiming to reduce the acquisition cost of \acrshort{ee} to be passed on to captive consumers ~\cite{modeloBR}.

% ``The Pool System would allow for the application of economies of scale in formulating the tariff for energy purchases; and a balanced allocation of risks and benefits of contracting (considered as a mechanism for captive consumers’ protection). The supply tariff would be the final average price based on the results of energy auctions. Distributors would only be able to purchase energy through the CCEE.
% The procurement auctions would solicit proposals to fulfil the energy supply or transmission needs listed in the plan. The proposal requiring the least revenue during the concession period would be the winning bid.
% To minimize electricity tariff increases, the new institutional model employs two types of electrical energy auctions: one for the generation of existing plants and the other for new plants. The winning bids in the former tend to be lower than those of the latter, thus contributing to lower the average prices.''~\cite{lud2007}

% \emph{Under the new rules, distributors are prohibited from generating their own power to supply their customers. Rather, they will have to bid for power from the pool system managed by CCEE.}

% In addition, the model in force requires, in both environments, the contracting of total demand by distributors and free consumers; new ballast calculation methodology
% \footnote{The ballast corresponds to the amount of \acrshort{ee} required to ensure consumption (in the case of consumers and distributors) or the sale (in the case of generators and merchants) of \acrshort{ee} in the commercial transactions that carry out ~\cite{gerusa}.}
% for sale of generation; contracting hydroelectric and thermoelectric plants in proportions that ensure a better balance between guarantee and cost of supply, as well as permanent monitoring of supply security ~\cite{modeloBR}.

% MERCADO SPOT
% Although the consumption of \acrshort{ee} is reasonably predictable by taking into account previous averages, consumer behaviour varies according to the demand of the financial market, climate, natural disasters, the development of new technologies and so on.
% In this way, there are energy differences regarding the purchase and sale contracts initially agreed upon, making necessary measures that guarantee the access and payment of \acrshort{ee} in a fair way by those who negotiate it \cite{ccee-market}.
% In the \acrshort{mcp}, also known as the spot market, these differences between the amounts generated, contracted and consumed are accounted for and settled.
% All contracts are necessary for the proper functioning of these operations managed in \acrshort{ccee}.


%================================================================================
\section{The worldwide electricity sector: general dynamics and forecasts}
\label{sec:world-sector}
% regulação internacionais no setor elétrico (ja temos algo nos artigos);

%----------------------------- Bom para vocabulário
% Understanding The World Of Electricity Trading
% https://www.investopedia.com/articles/investing/042115/understanding-world-electricity-trading.asp
% 24/05/2018
%-----------------------------

Electricity Market and Renewables~\cite{erik2017} --> good text to link between actual market and change ones (different market designs)

``The main objective of the reform was to allow government to focus on its role as policy-maker and regulator, transferring the responsibility of operations and investment to the private sector. The Brazilian government was following a world tendency, which is a government focused on regulation, minimizing its roles on the economy and encouraging the private investment''~\cite{lud2007}.

% \cite{lud2007}
% The regulation activities from the regulatory agencies are made, basically, in three areas, i) economic regulation, ii) technical and customer service regulation, and iii) facilitation of competition.

% HOW IS IT STRUCTURE? STAKEHOLDERS, REALMS AND INFRASTRUCTURE...

All the time electricity sector comes up on a discussion, all sort of ``depends of'' rises up.
Basically, the sector operation intricacy contribute to it.
Classification of players by voltage levels and power consuming, or definition of stakeholder classes into generation, transmission and so on, is some of the challenges when talking about sector behavioral model.

With objective to better explain the relationship between stakeholders and all factors that involve them, we suggest the \autoref{fig:stakeholder} to guide a general view of one's position on the sector.
%Influenced by the concept of power flexibility presented by IEA~\cite{iea2011}, which it refers to ...
The base role of each stakeholder is considered by its dynamics relating to four realms:
\begin{itemize*}[font={\bfseries}, itemjoin={{; }}, itemjoin*={{; and }}, afterlabel={{, }}]
  \item [market] gathering financial and economic subjects
  \item [operation] planning and management of all energy chain to keep electricity flowing
  \item [hardware] assembling all sort of equipment, from measurement to protection, from power lines to communication
  \item [policy and regulations] involving government strategies and standards.% security subjects is implicit in standard
\end{itemize*}

The first three are core on how stakeholders interact.
For instance, consider a main stakeholder to analyze (placing it on the center of the diagram), its relationships with other ones are represented by those three realms intersections (R$_{1}$, R$_{2}$ and R$_{3}$).
% R is for relationship, but looks like it`s for realm... CHANGE IT!
The forth realm, although differently interacting for each overlap, is hidden at a first look, but present on the background to keep relationships working fine.

Identify stakeholders by name is useful to clarify the diagram (already presented on \autoref{sec:intro}).
So, normally, they are classified by its role on the sector, which implies in
\begin{itemize*}[label={}, before=\unskip{:}, itemjoin={{,}}, itemjoin*={{ and}}]
    \item \textbf{generator}
    \item \textbf{transmitter}
    \item \textbf{distributer}
    \item \textbf{trader}
    \item \textbf{consumer}%
    %\item \textbf{prosumer};% only for RESIDENTIAL?
    \item \textbf{regulator}% MEDIATOR?
\end{itemize*}
[missing reference].
Even aware of subclassifications, concerning by power and voltage levels, we only dismember consumer into residential, commercial and industrial [missing reference]. % WHY?
% Furthermore, 

Back to \autoref{fig:stakeholder}, let's analyze an example on how a generator behaves.
Its R$_{1}$ refers to whom it's sending power. This peer can be a transmitter, a trader, a distributer or a industrial consumer.
The R$_{2}$ refers to how the power will be evaluated, depending on price and period trading. It's attached with R$_{1}$ consequence.
And the R$_{3}$ refers to physical connection established between generation machines and transmission lines, which results on the transmitter role solely.
% Even it looks like a charm, so simple, clean and wise.

\refimage{\uppercase{stake\\holder}}{\uppercase{hardware}}{\uppercase{market}}{\uppercase{operation}}{R$_{3}$}{R$_{2}$}{R$_{1}$}{\uppercase{policy \& regulations}}{Statement of any stakeholder on electricity sector regarding its realms and relationships (R$_{n}$).\label{fig:stakeholder}}

This system can be repeated for each stakeholder without lose its relationship dynamics...
At first glance, the diagram solves the intricacy of electric sector behavior drawing side by side each stakeholder's point of view of the system and interconnecting its dependencies. % As shown on Figure 2.
Furthermore, its helps visualize % similarities and disparities [if talk about it, must give examples...]
how each stakeholder behaves under specific policy and regulations. % by country [if talk about it, must give examples...]

% [missing paragraph unifying structure and market behavior]
Notwithstanding, those realms play crucial value when considering technology developments, although how its implemented in the sector depends on factors...

In this way, as stated by \citeauthor{book:networks}~\cite{book:networks}:
\begin{quotation}
``[\dots] the unique characteristics of electricity industries appeared to set them apart from most other industries, deemed `competitive'.
These electricity industries notably feature:
\begin{itemize*}[font={\it}]
  \item [significant economies] of scale or scope (extending to natural monopolies);
  \item [far-reaching externalities] (positive or negative) in production or consumption;
  and \item [extensive vertical and horizontal integration] (either under a single corporate umbrella or in the form of long-term ad hoc contracts).
\end{itemize*} --- {\it \small (emphasized by authors)}
\end{quotation}

\medskip % TRY ON MARKET SIDE
Nonetheless, on \emph{Technology Roadmap: Smart Grids}~\cite{iea2011} is discussed that outmoded regulatory and market models can slow down power grid development towards a smart one, as they usually don't follow new technology approaches on the same pace its launched.
Its complements with pros and cons, summarized on \autoref{tab:unbundling}, about
``vertically integrated utilities, which [stakeholders] own and operate infrastructure assets across the generation, distribution and transmission sectors'' and
horizontally ones, ``which is intended to allow increased competition [working by] market-based and regulated units''.

\begin{table*}[!t]
  % increase table row spacing, adjust to taste
  \renewcommand{\arraystretch}{1.3}
%   if using array.sty, it might be a good idea to tweak the value of
%   \extrarowheight as needed to properly center the text within the cells
  \caption{Comparison of electricity market operation options.}
  \label{tab:unbundling}
  \centering
  % Some packages, such as MDW tools, offer better commands for making tables
  % than the plain LaTeX2e tabular which is used here.
  \begin{tabular}{| m{0.2\textwidth} | m{0.35\textwidth}| m{0.35\textwidth} |}
    \hline
    ATTRIBUTES & VERTICAL & HORIZONTAL\\
    \hline
    costs and benefits from the deployment of technology & shared and managed efficiently across stakeholders & difficult to capture on a system-wide basis \newline (especially with respect to smart grids)\\
    investment and development & fully integrated & segregated by stakeholders\\
    competition & difficult, which could hinder innovation and increase prices for consumers \newline (however, it depends largely on whether the market is governed by appropriate regulatory structures) & conducted by market-based activities\\
    stakeholders engagement & bundling of generation, distribution and transmission roles & typically include generation and retail roles on market rules \newline and transmission and distribution under regulated activities\\
    some benefits & velocity in deploy investments & primarily a continued downward pressure on prices\\
    overall complexity & usual & higher\\
    \hline
    \multicolumn{3}{r}{\scriptsize{Based on \emph{Technology Roadmap: Smart Grids}~\cite{iea2011}}.}\\
  \end{tabular}
\end{table*}




%--------------------Como resumir isso---------------------------IEA
% In the generation sector, markets have developed in which generators sell electricity within a structure defining prices, time frames and other rules.
% In the retail sector, sometimes the distribution system operator still retails the electricity to consumers and sometimes new participants enter the market that sell only electricity services.
%--------------------Como resumir isso---------------------------IEA

% SECTOR RE-STRUCTURING ... WHY? --> tese ANEEL \cite{lima:2006} --> DA FORMAÇÃO DE UM “MERCADO” DE ENERGIA ELÉTRICA
\begin{quotation}
Within this very specific framework, the successful introduction of competitive mechanisms, substituting for administered regulation or internal corporate management hierarchies, along with the creation of open markets either upstream or downstream of the formerly integrated networks, created both disruptions and innovations in equal measure.''~\cite{book:networks}
\end{quotation}

\medskip % TRY ON OPERATION & HARDWARE SIDEs
% WHEN SMART GRID WAS DEFINED? ---WHY?---- IS IT A CONTINUATION OF PREVIOUSLY RE-STRUCTURING?
The barriers faced in unbundling the electricity sector was surpassed by smart grids concept [myref - procurar],
``the infrastructure that enables the delivery of power from generation sources to end-uses to be monitored and managed in real time''~\cite{iea2011}.
Although
\begin{quotation}
Most current smart grid pilot projects focus on network enhancement efforts such as local balancing, demand-side management (through smart meters) and distributed generation.
Demonstration projects have so far been undertaken on a restricted scale and have been hindered by limited customer participation and a lack of a credible aggregator business model.
Data (and security) challenges are likely to increase as existing pilots expand to larger-scale projects.
Non-network solutions such as ICTs are being used in a growing number of smart grid projects, bringing a greater dependence on IT and data management systems to enable network operation~\cite{iea2011}.
\end{quotation}

\citeauthor{iea2011} highlights on the roadmap predictions that smartening the grid until 2050 will require the correlation of current stakeholders with the financial sector, research and international organizations.

% WHY DISTRIBUTED?
The integration of distributed generation has encourage discussions about electricity sector course [ref].
On the beginning, many claims on grid stability and integrity (reliability) [ref].
Nowadays, more acceptability and studies to deal with needed changes [ref-IEEE-PES].
However,% talk about NON innovative pattern of electricity sector
%Influenced by inertia mechanisms to change traditional business model towards technology development out of the hardware scope, which implies simply on energy efficiency [myref].

%----------------------------------------------------------------------
Enumerate key elements to a competitive strategy...

They've got the following result
"Thus, the building of competitive markets combines three main dimensions:
(1) the overall separation of potentially competitive activities from inherently monopolistic electricity activities (unbundling);
(2) the segregation of all the operations and transactions of the industry into modules organized around various mechanisms for internal coordination (modularity); and
(3) the implementation of the various modules in the chain to carry the competitive transactions (sequentiality). The following subsections explain the theoretical foundation of electricity market reforms and provide an empirical illustration for them"

Being aware about these,
"In this section the objective is to highlight the main remaining issues in the electricity industry around the world, namely,
the question of investment in electricity markets,
the treatment of long term contracts and
the support of Renewable Energy Sources for Electricity (RES- E) in energy markets"
% make correlation with proposed model
%----------------------------------------------------------------------



% WHY MINI/MICRO GRID (CUSTUMER FOCUSED)


% HIGHLIGHTS OF ACHIEVEMENTS AND EXPECTATIONS

However into the market realm exist two sub-domains, known as opened and closed market.
The first constitutes of free negotiation of power price between stakeholders.
While the second pays power price by a fix rate, previously defined by its seller under government statements.

For instance, residential customers represent a class of stakeholders that usually has merely one relationship , with distribution utility, to deal with all aspect of electricity sector.
Differently is the case of industrial customers that can be in touch directly with generators ($R_{1}$ e $R_{3}$) to transact ... and with distribution utility to guarantee that power arrive at consuming point ($R_{2}$).

%In addition, 
%Nevertheless,
%However,
%Although,


% UNBUNDLING = DISTRIBUTED?


Concluding with changes \dots \cite{sven2013}

%================================================================================