%\usepackage[brazil,american]{babel}	% adequação para o português Brasil
\usepackage[brazil,UKenglish,british]{babel}	% adequação para o português Brasil
\usepackage[utf8]{inputenc}		    % Determina a codificação utilizada
								    % (conversão automática dos acentos)
\usepackage{makeidx}			    % Cria o índice
\usepackage{hyperref}			    % Controla a formação do índice
\usepackage{indentfirst}	        % Indenta o primeiro parágrafo de cada seção.
\usepackage{graphicx}			    % Inclusão de gráficos
\graphicspath{
    {pics/}
}
\usepackage{subfig}
\usepackage{amsmath}			    % pacote matemático
\makeatletter
\@fleqnfalse
\@mathmargin\@centering
\makeatother
\usepackage[inline]{enumitem}	    % lista no parágrafo
\usepackage{tikz}				    % awesome drawings
\usetikzlibrary{calc}               % to get node width and height
\usetikzlibrary{positioning}        % to use the relative positioning commands
\usetikzlibrary{arrows.meta}
\usetikzlibrary{mindmap}
\usetikzlibrary{shapes.geometric}   % to use diamond shape
\usepackage[table]{xcolor}		    % colors on tables
\definecolor{lightgray}{gray}{0.9}
% \let\oldtabular\tabular             % tabela cor-sim/cor-não
% \let\endoldtabular\endtabular
% \renewenvironment{tabular}{\rowcolors{2}{white}{lightgray}\oldtabular}{\endoldtabular}
\usepackage{multirow}
\usepackage{pgffor} 				% required for \foreach
                                    % (loop statement in LaTeX)
\usepackage{chngpage}			    % allows for temporary adjustment
                                    % of side margins
\usepackage{titlecaps}              % required for \titlecap
\usepackage{listings}               % for codes environment
% \lstloadlanguages[sharp]{C}
\lstset{defaultdialect=[sharp]C}

\lstdefinestyle{mystyle}{
    % backgroundcolor=\color{backcolour},   
    % commentstyle=\color{codegreen},
    % keywordstyle=\color{magenta},
    % numberstyle=\tiny\color{codegray},
    % stringstyle=\color{codepurple},
    basicstyle=\scriptsize,
    breakatwhitespace=false,         
    breaklines=true,                 
    captionpos=b,                    
    keepspaces=true,                 
    numbers=left,                    
    numbersep=5pt,                  
    showspaces=false,                
    showstringspaces=false,
    showtabs=false,                  
    tabsize=2,
    frame=lrtb
}
\lstset{style=mystyle} % https://pt.overleaf.com/learn/latex/Code_listing

% ---
% Pacote auxiliar para as normas da UERJ
% ---
\usepackage[frame=no,algline=yes,font=default]{config/repUERJformat}
\usepackage{config/repUERJpseudocode}
% ---
% Pacotes de citacoes
% ---
\usepackage[alf]{abntex2cite}

% ---
% Configuração do glossário
% ---

% \usepackage{config/siglas-poliglota}

% facilita a criação do glossário
\usepackage[acronym, nonumberlist, nopostdot]{glossaries}
% \renewcommand*{\glsclearpage}{}     % organização do glossário em ordem
                                    % alfabética, independe da ordem que
                                    % os dados são inseridos
                                    
\addto\captionsamerican{%
    \renewcommand{\acronymname}{List of abbreviations and acronyms}%
    \renewcommand*{\glossarysection}[2][]{\pretextualchapter{#2}}%
    % \renewcommand*{\glossarysection}[2][]{%  
    %     \twocolumn[{\pretextualchapter{#2}}]%  
    %     \setlength\glsdescwidth{0.6\linewidth}%  
    %     \glsglossarymark{\glossarytoctitle}%  
    % }
    % \renewcommand*{\glossarypostamble}{\onecolumn}
}

\addto\captionsbrazil{%
    \renewcommand{\acronymname}{\abrevname}%
}

\newacronym{aneel}{ANEEL}
% {Agência Nacional de Energia Elétrica}
{Brazilian Electricity Regulatory Agency}

\newacronym{ccee}{CCEE}
% {Câmara de Comercialização de Energia Elétrica}
{Electric Power Trading Chamber}

\newacronym{epe}{EPE}
% {Empresa de Pesquisa Energética}
{Energy Research Company}

\newacronym{mme}{MME}
% {Ministério de Minas e Energia}
{Brazilian Ministry of Mines and Energy}

\newacronym{sin}{SIN}
% {Sistema Interligado Nacional}
{National Power Grid System}

\newacronym{cmse}{CMSE}
% {Comitê de Monitoramento do Setor Elétrico}
{Power Sector Monitoring Committee}

\newacronym{ons}{ONS}
% {Operador Nacional do Sistema Elétrico}
{National Power System Operator}

\newacronym{acr}{ACR}
% {Ambiente de Contratação Regulada}
{Regulated Contracting Environment}

\newacronym{acl}{ACL}
% {Ambiente de Contratação Livre}
{Free Contracting Environment}

\newacronym{seb}{SEB}
% {Setor Elétrico Brasileiro}
{Brazilian Electrical Sector}

\newacronym{iaas}{IaaS}
% {Infraestrutura como um serviço}
{Infrastructure as a Service}

\newacronym{paas}{PaaS}
% {Plataforma como um serviço}
{Platform as a Service}

\newacronym{saas}{SaaS}
% {Software como um serviço}
{Software as a Service}

\newacronym{baas}{BaaS}
% {Blockchain como um serviço}
{Blockchain as a Service}

\newacronym{reseb}{Projeto RE-SEB}
% {Projeto de Reestruturação do Setor Elétrico Brasileiro}
{Restructuring of the Brazilian Electricity Sector Project}

\newacronym{der}{DER}
% {Recursos Energéticos Distribuídos}
{Distributed Energy Resources}

\newacronym{gd}
% {GD}{Geração Distribuída}
{DG}{Distributed Generation}

\newacronym[plural=DLTs]{dlt}{DLT}% ,firstplural=Distributed Ledger Technologies (DLTs)
% {Tecnologia de Registros Distribuída}
{Distributed Ledger Technology}

\newacronym[plural=Dapps,firstplural=distributed applications (Dapps)]{dapp}{Dapp}
% {aplicação distribuída}
{distributed application}

\newacronym{Dapp}{Dapp}
% {Aplicação Distribuída}
{Decentralized Application}

\newacronym{p2p}{P2P}
% {}
{peer-to-peer}

\newacronym{mcp}{MCP}
% {Mercado de Curto Prazo}
{}

\newacronym{pis}{PIS}
% {Programas de Integração Social e de Formação do Patrimônio do Servidor Público}
{}

\newacronym{cofins}{COFINS}
% {Contribuição para Financiamento da Seguridade Social}
{}

\newacronym{icms}{ICMS}
% {Imposto sobre Circulação de Mercadorias e Serviços}
{}

\newacronym{ee}{EE}
% {Energia Elétrica}
{Electric Energy}

\newacronym{fv}{FV}
% {Energia Elétrica Fotovoltaica}
{}

\newacronym{pv}{PV}
% {}
{Photovoltaic}

\newacronym{proinfa}{PROINFA}
% {Programa de Incentivo às Fontes Alternativas de Energia Elétrica}
{}

\newacronym{its}{ITS Rio}
% {Instituto de Tecnologia e Sociedade do Rio}
{}

\newacronym{b2b}{B2B}
% {}
{business-to-business}

\newacronym{c2c}{C2C}
% {}
{consumer-to-consumer}

\newacronym{b2c}{B2C}
% {}
{business-to-consumer}

\newacronym{iot}{IoT}
% {}
{Internet Of Things}

\newacronym{tic}{TIC}
% {Tecnologias da Informação e Comunicação}
{}

\newacronym[plural=ICTs, firstplural=Information and Communications Technologies (ICTs)]{ict}{ICT}
% {Tecnologia da Informação e Comunicação}
{Information and Communications Technology}

\newacronym{pow}{PoW}
% {}
{Proof of Work}

\newacronym{pos}{PoS}
% {}
{Proof of Stake}

\newacronym{ico}{ICO}
% {}
{Initial Coin Offering}

\newacronym{poet}{PoET}
% {}
{Proof of Elapsed Time}

\newacronym{pob}{PoB}
% {}
{Proof of Burn}

\newacronym{por}{PoR}
% {}
{Proof of Retrievability}

\newacronym{poa}{PoA}
% {}
{Proof of Authority}

\newacronym{tee}{TEE}
% {}
{Trusted Execution Environment}

\newacronym{pbft}{PBFT}
% {}
{Practical Byzantine Fault Tolerance}

\newacronym{sbft}{SBFT}
% {}
{Simplified Byzantine Fault Tolerant}

\newacronym{dbft}{DBFT}
% {}
{Delegated Byzantine Fault Tolerance}

\newacronym{mtemsm}{MTEMsm}
% {}
{Microgrid Transactive Energy Management Smart Contract}

\newacronym{pp}{\emph{PP}}
% {}
{Power Plant}

\newacronym{tg}{\emph{TG}}
% {}
{Token Generator}

\newacronym{rum}{\emph{Rum}}
% {}
{Referendum}

\newacronym{mem}{\emph{Member}}
% {}
{Member}

\newacronym[plural=APIs, firstplural=Application Programming Interfaces (APIs)]{api}{API}
% {Interface de Programação de Aplicações}
{Application Programming Interface}

\newacronym{uml}{UML}
% {Linguagem de Modelagem Unificada}
{Unified Modeling Language}

\newacronym{te}{TE}
% {Energia Transativa}
{Transactive Energy}
				    % lista de acrônimos
% \newcommand{\gloss}[3]{\newglossaryentry{g#1}{name={#2},description={#3}}} % 'g#1' is the glossary label
% % General glossary input in the form {'name'}/{'description'}
% TAXONOMY --> XU2017
\def \listofBCc {
				{application}/{diferenciação para o modelo OSI},
% 
                {architecture}/{structure?},
%                 
				{block}/{containers?},
%                 
                {blockchain}/{"The term “blockchain” is used to refer to a data structure and occasionally to a network or system.
                As a data structure, a blockchain is an ordered list of blocks, where each block contains a small (possibly empty) list of transactions. Each block in a blockchain is “chained” back to the previous block, by containing a hash of the representation of the previous block. Thus historical transactions in the blockchain may not be deleted or altered without invalidating the chain of hashes. Combined with computational constraints and incentive schemes on the creation of blocks, this can in practice prevent tampering and revision of information stored in the blockchain."~\cite{xu2017}},
% 
                {blockchain network (DLT)}/{},
%                 
                {computação quântica}/{},
% 
                {consensus}/{"A signed transaction is sent to a node connected to the blockchain network, which validates the transaction. If the transaction is valid and previously unknown to the node, the node propagates it to other nodes in the network, which also validate the transaction and propagate it to their peers, until the transaction reaches all nodes in the network. In a global network, this can take seconds. There are different consensus mechanisms, e.g., “proof-of-work” or “proof-of-stake” (see Section III). Depending on the consensus mechanism and the required guarantees, there can be different notions of when a transaction is taken to be committed or confirmed and thus immutable."~\cite{xu2017}},
% 
                {hash}/{},
%                 
                {client-server}/{},
%                 
                {chaining}/{pq esse termo?},
% 
                {computação quântica}/{},
%                 
                {control data redundancy}/{"In the Database approach, ideally each data item is stored in only one place in the database. In some cases redundancy still exists so as to improve system performance, but such redundancy is controlled and kept to minimum."},
%                 http://www.opentextbooks.org.hk/ditatopic/30651		30/10/18
% 
                {crypto-currency}/{},
%                 
                {cryptography}/{},
%                          
                {DAO}/{},
%                 
                {dapps}/{},
%                 
                {data structure}/{ordered list of blocks?},
% 
                {double-spending problem}/{},
% 
                {ledger}/{},
%                 
                {Merkle Tree}/{árvore de armazenamento de dados no banco de dados...?},
% 
                {mining/mined}/{"Mining is the process of appending new blocks to the blockchain data structure. A blockchain network relies on miners to aggregate valid transactions into blocks and append them to the blockchain. New blocks broadcast across the whole network, so that each node holds a replica of the whole data structure. The whole network aims to reach a consensus about the latest block to be included into the blockchain."~\cite{xu2017}},
% 
                {minted}/{},
% 
                {network}/{centralized or decentralized?},
%                 
                {node}/{},
%                 
                {peer-to-peer (P2P)}/{},
%                 
                {protocol}/{},
% 
                {private key}/{},
% 
                {public key}/{"Public key cryptography and digital signatures are normally used to identify accounts and to ensure authorization of transactions initiated on a blockchain."~\cite{xu2017}},
%                 
                {relational database}/{},
%                 
                {server-client}/{},
%                 
                {signature}/{},
%                 
                {smart contract}/{"Programs can be deployed and run on a blockchain, and are known as smart contracts [18]. Smart contracts can express triggers, conditions and business logic [25] to enable more complex programmable transaction. However, they are not necessarily smart, nor necessarily related to legal contracts. A common simple example of a smart contract-enabled service is escrow, which can hold funds until the obligations defined in the smart contract have been fulfilled."~\cite{xu2017}},% E se a pessoa não tiver como pagar no futuro? Isso não está igual a lógica do cheque?
% 
                {``stack-oriented'' programming}/{},
%                 
                {transaction}/{"Transactions are data packages that store parameters (such as monetary value in the case of Bitcoin) and results of function calls (such as from smart contracts). The integrity of a transaction is checked by algorithmic rules and cryptographic techniques. A transaction is signed by its initiator, to authorise the expenditure of their money, to authorise the data payload of a transaction, or the creation and execution of a smart contract."~\cite{xu2017}},
% 
                {tamper-proof}/{},
% 
                {timestamp}/{},
%                 
%                 {}/{},
%                 {}/{},
%                 {}/{},
%                 {}/{},
% 
                {z}/{o último sem vírgula}
				}


% Adjust the glossary inputs to glossary package code to allow sorting
\newcounter{BC}
\foreach \name/\description in \listofBCc{
    \refstepcounter{BC}
	\gloss{\theBC}{\name}{\description}
}				  % blockchain glossary
% % Smart Grid + VPP + Energy Internet + Transactive Grid + SEN (1 parágrafo para cada assunto + 1 parágrafo conectando eles)

\section*{Smart Grid}


\section*{Virtual Power Plant}


\section*{Energy Internet}


\section*{Transactive Energy}
% Transactive energy (TE) can be defined as “a system of economic and control mechanisms that allows the dynamic balance of supply and demand across the entire electrical infrastructure using value as a key operational parameter.”  This definition, which is the one currently used by NIST, was originally proposed by the U.S. Department of Energy’s Gridwise Architecture Council in its Transactive Energy Framework (link is external).
% %https://www.nist.gov/engineering-laboratory/smart-grid/transactive-energy-overview

% The term "transactive energy" is used here to refer to techniques for managing the generation, consumption or flow of electric power within an electric power system through the use of economic or market based constructs while considering grid reliability constraints. The term "transactive" comes from considering that decisions are made based on a value. These decisions may be analogous to or literally economic transactions. An example of an application of a transactive energy technique is the double auction market used to control responsive demand side assets in the GridWise 
% %https://www.gridwiseac.org/about/transactive_energy.aspx

\section*{Smart Energy Network}


% GENERAL CONSEQUENCES AND GAPS                  % smart-grid glossary
\makeglossaries                     % unsorted output?
% \makenoidxglossaries                % sorted output?

% Lista customizada
% inline list with bold label with items separated by semicolon
\newenvironment{descriptive}{\begin{itemize*}[font=\bfseries, before=\unskip{: }, itemjoin={{; }}, itemjoin*={{; and }}, afterlabel={{ -- }}]}{\end{itemize*}}

% Acrônimo customizado
\newcommand{\glsitem}[1]{\glsreset{#1}\gls{#1}}

% Coloca a primeira letra da sentença em maiúscula
\let\orgautoref\autoref
\providecommand{\Autoref}[1]{%
    \def\appendixautorefname{Appendix}%
    \def\sectionautorefname{Section}%
    \def\chapterautorefname{Chapter}%
    % \def\postextualchapterautorefname{Appendix}%
    \orgautoref{#1}%
}

\def\copyright{\UERJautornome\space\UERJautorsobrenome\space\textcopyright\space \href{http://creativecommons.org/licenses/by/4.0/}{CC BY 4.0}}

% Template do fluxograma
\tikzset{%
    >={Latex[width=2mm,length=2mm]},
    % Specifications for style of nodes:
            base/.style = {rounded corners, draw=black, yshift=-1cm,
                           minimum width=2cm, minimum height=1cm, text width=3cm, text centered},
  activityStarts/.style = {ellipse, base, text width=2cm, fill=blue!30},
       startstop/.style = {diamond, base, text width=1.5cm, fill=red!30, node distance=2.5cm},
    activityRuns/.style = {rectangle, base, fill=green!30},
         process/.style = {rectangle, base, fill=orange!15},
}

\newcommand{\flowchart}[4][h!tbp]{%
    \begin{figure}[#1]{\textwidth}
    	\centering
        \frame{%
        %     \resizebox{\textwidth}{!}{%
            % Drawing part, node distance is 1.5 cm and every node is prefilled with white background
            \begin{tikzpicture}[every node/.style={fill=white,
                                                    font=\sffamily\scriptsize,
                                                    node distance=1cm},
                                align=center]
                {#4}
        	\end{tikzpicture}%
        }
    	\caption{#2}
    	\label{#3}
    % 	\legend{}
        \source{\copyright}
    \end{figure}
}

\newcommand{\figchart}[4][h!tbp]{\flowchart[#1]{#2}{#3}{#4}}

% Imagem sobre stakeholders
\newcommand{\refimage}[9]{

\begin{figure}[h!tb]{\textwidth}
	\caption{#9}
    \centering
    
    % definition of circles
    \def\fcircle{(90:1.2) circle (2)} % HARDWARE
	\def\scircle{(210:1.2) circle (2)} % MARKET
    \def\tcircle{(330:1.2) circle (2)} % OPERATION
    
    %definition of image background
    \def\sil{3.5} % (sil = Square Image Limit)
    \def\bcksqr{\filldraw[fill=gray!10, draw=gray!80, thick, rounded corners=0.2cm] (-\sil,-\sil) rectangle (\sil,\sil)} % POLICY AND REGULATION
    
    % Venn diagram
    \begin{tikzpicture}
      % background
      \bcksqr;
      % circles area
      \draw[fill=white] \fcircle;
      \draw[fill=white] \scircle;
      \draw[fill=white] \tcircle;
      % R_1
      \begin{scope}
        \clip \fcircle;
        \draw[draw=none, fill=gray!35] \scircle;
      \end{scope}
      % R_2
      \begin{scope}
        \clip \scircle;
        \draw[draw=none, fill=gray!35] \tcircle;
      \end{scope}
      % R_3
      \begin{scope}
        \clip \fcircle;
        \draw[draw=none, fill=gray!35] \tcircle;
      \end{scope}
      % STAKEHOLDER
      \begin{scope}
        \clip \fcircle;
        \clip \scircle;
        \fill[fill=gray!60] \tcircle;
      \end{scope}
      % circle edge
      \draw[draw=gray!80, thick] \fcircle;
      \draw[draw=gray!80, thick] \scircle;
      \draw[draw=gray!80, thick] \tcircle;
      % tag names
      \node [align=center]	{#1};
      \node at (0,2)		{#2};
      \node at (-2,-0.75)	{#3};
      \node at (2,-0.75)	{#4};
      \node at (1.25,0.75)	{#5};
      \node at (0,-1.5)		{#6};
      \node at (-1.25,0.75) {#7};
      \node at (0,-3.15)	{#8};
    \end{tikzpicture}
    
%     \legend{Texto da legenda.}
% 	\source{Citação da fonte ou `O autor.'.}
\end{figure}

}

% Configuração padrão para os fluxogramas
\tikzset{%
    >={Latex[width=2mm,length=2mm]},
    node distance=1.5cm,
    % 
    % Specifications for style of nodes:
              base/.style = {fill=white, draw=black, %yshift=-1cm,
                             %minimum width=2cm, minimum height=1cm,
                             text width=2.5cm, text centered, font=\sffamily},
           onchain/.style = {base, rectangle, rounded corners},
          offchain/.style = {onchain, dashed},
          terminal/.style = {base, ellipse},
               i-o/.style = {base, trapezium, trapezium left angle=60, trapezium right angle=120},
               i2o/.style = {i-o, text width=5cm},
          decision/.style = {base, diamond},
             erase/.style = {fill=white, shape=circle, minimum size=1.8*\radius, inner sep=0pt},
}

% VALUABLE NOTES
% \draw[->]     (d) -- ++(-4cm,0) |- (end) node[pos=0.25] {No};
% [node distance=1.5cm, align=center]

\def \radius {.7mm} 
\def \cc {\textcopyright\space\textsc{CC BY 4.0}\space\textsc{2019, Yuri Bastos Gabrich}}
