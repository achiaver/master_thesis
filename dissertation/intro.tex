\href{http://blockchain.mit.edu/}{\Huge http://blockchain.mit.edu/}

The worldwide growing demand for electricity is not solely a consequence of the population growth but by the rising of their well-being and quality of life~\cite{UN-SDG, IEA2017}.
Heavily influenced so far by the technological development, the energy base required to sustain these electricity demands is, economic and environmentally, unsustainable.
Since the ending of the industrial era great advent, 
developed and under-developing nations are experiencing, on different paces, the switch from non-renewable, highly concentrated and centralized energy units with non-diversified power sources to alternative methods to handle with this energy issue~\cite{toffler1980}.

More than ever, a variety of solutions is being experienced at every part of the world.
Until recently, most of them keeps the conservative infrastructure and market models, just replacing the power source fuels by renewable ones, as the case with solar and wind energy for example.
They continue to be big power plants built up either on- and offshore, depending on each country land space and energy reservoir potential.
Independently of the solution type, the challenge to assure electricity for each individual at a reasonable cost, respecting personal conditions, and being sustainable friendly is no small task.

However, the technological development also brings another special feature, the products price and size shrinking.
This gives room to further discussions about the current business models towards market alternatives that had relied only on the imagination of few's.
For instance, the \gls{gd} of electricity has been increasingly implemented to supply local power demands due to accessible costs of \gls{der} [tem uma ref do IEEE do ano passado sobre isso!].
Since big power plants are out of the scope to support the before-mentioned demands either because of infrastructure cost or environmental concerns~\cite{epe2014-2, castro2018},
it is expected a more active participation of end power consumers into the energy base generation,
which may influence on the economy and in the structure of the electricity sector [todas as refs do IEEE...].

% 1. What is transactive energy
A new electricity market design to support all this upcoming changes is been referenced as \gls{te}~\cite{olken2016}.
A system approach to facilitate the integration of an amount of \glspl{der},
to provide transparent energy prices, and
to allow power consumers of all sizes to trade energy~\cite{forfia2016}.
For instance, end customers can now choose to trade with different power sources based on generation type, sustainable factors, and anything else over the average cost-based of the electricity price formulation.

A concept that resembles the sector restructuring discussion from the unbundling of vertically integrated utilities to open the electricity market until low-voltage consumers but impracticable due to the technology of that time.
Nowadays, this purpose arises from another perspective, with much more caution and business model options, besides the one-way solution of centralized control and operation of the power system.
It is also known that the current top-down model is not appropriate to allow local market incentives on the forthcoming micro/mini-grid coordination efforts to benefit its users in a transparent manner~\cite{kamphuis2008}.

% 1.1 Why this exist?
Bearing this in mind,
there is a continuous general effort for social integration into the renewable energy segment by means of the micro/mini-grid~\cite{Jiayi2008}.
A small scale and resilient electric grid connects the \glspl{gd} with end consumers throughout the existing distribution grid, independently of the governance used to rule this relationship.
There is a belief that this model
"will bring a dynamic clean energy economy that empowers communities and customers — across all income levels, geographies, and demographics — to take control of their energy use, driving local economic growth and revitalization, improving the resiliency of our energy system, and protecting our environment."~\cite{masiello2016}

% 1.2 What are the context in Brazil and Worldwide?
Since 2010's decade, several reports have been conducted to state challenges and plannings to implement such generations.
For instance, some of them are
the \emph{Technology Roadmap: Smart Grids}~\cite{iea2011},
the \emph{Insertion of Distributed Photovoltaic Generation in Brazil - Conditioners and Impacts}~\cite{epe2014}, and
the \emph{Energy Access Outlook 2017, from poverty to prosperity}~\cite{IEA2017}.

Nonetheless, the discussions bump on different points of views about how market and business models should operate.
Electric technical issues used to be the claims against a distributed management and operation of the grid because it could create undesirable constraints on the electricity quality and reliability,
and therefore impact on the electricity price definition and so forth.
However, some alternatives have been arising with the integration improvements of \gls{ict} with power systems.
% Isso é smart grid!

\citeonline{kamphuis2008} had identified four fundamental issues to implement \gls{ict} in power systems, which concerns with
\begin{itemize*}[label=(\roman*), itemjoin={{; }}, itemjoin*={{; and }}]
    \item system architecture to support the coordination between algorithm and physical control mechanism
    \item scalability
    \item operational planning to keep energy demand flexibility due to the huge intermittent power sources to consider
    \item timing discrepancy of the communication process between planning and real-time operation (because price is defined on forecast studies and with so much \gls{gd} the complexity rockets :P)
\end{itemize*}
In addition, the mentioned author had suggested the use of a \gls{p2p} network, instead of the client-server model, to create a useful system to handle with the communication challenge
% of implementing real-world mappings of grid coordination strategies...??
% DETALHAR O QUE kamphuis fez.

"Future implementation of event-based techniques will allow implementation of even more real-world mappings of grid coordination strategies."
\citeonline{kamphuis2008}




% 2. Why blockchain is related to this?
In this context, the blockchain has the basic requirements to integrate several power customer profiles into an unique informational network system able to exchange data about power generation, consumption and transactions without compromising the core business roles of the power utilities and the power customers, i.e., keeping the responsibility of the former to manage the electricity distribution and providing the best electricity service for the latter.
On the other hand, the blockchain application must comply with local jurisdiction in order to be relevant, even with the technical specifications about \gls{gd} being almost the same worldwide.

% 2.1 What is blockchain?
The blockchain technology has been considered a revolution in the way the Internet is used.
However, this view reminds the same feeling at the time the Internet itself began to become popular~\cite{book:internet}.
% As introduced by~\citeonline{book:internet}, % trocar o "e" por "and"
% the Internet is an % nothing more than an
% \begin{quote}\it
% ``international conglomeration of computer networks [which] has been seen as a way of revolutionizing the relationship between people.
% New ways of communicating, of acting, of finding knowledge, of behaving [\dots]
% Overcoming the barrier of distance quickly and relatively cheaply, ended up creating new and different opportunities for work and leisure.
% The Internet is also seen as an anarchy, a world without borders, the future in the present.
% But it is not an isolated world from the one in which we live: it influences and is influenced by it.''
% \end{quote}

In this way, understanding how blockchain fits into the Internet context is critical to develop an application that is compatible with it.
By using already established technologies such as \gls{p2p} network architecture and cryptographic keys,
we can understand the blockchain concept and the differences between those technologies without delving into their fundamentals.

The blockchain innovates in the type and in the form that communication occurs among its users, in which the transmitted data represents a certain information, but not the information content itself.
%Essa ultima frase está linda. Parabéns! (VITOR)
In addition, the veracity and accessibility are ensured by the controlled redundancy of data,
% http://www.opentextbooks.org.hk/ditatopic/30651
after it has been verified by specific users of the blockchain network.
% in the blockchain network, not in the application network! (será q é assim para todas as plataformas?)

Thereby, the blockchain technology represents a new view on the current business management model like the Internet had made several decades ago.
Its applications have the potential to overcome saturated business models and to promote new ones, such as the case with the electricity sector, which has been experiencing an expansion of types, sizes and units of power generation.

% 2.2 Are there someone else doing it?
In Brazil, the government is aware that electric power is a common good, as well as about its importance in the daily life of society, whether in residences or in various sectors of the economy.
%Tem um conectivo que está mudando o sentido no meu entendimento, talvez seja esse to remunerate
For this reason, the electricity demands fair tariffs
% to remunerate its infrastructure in order
to maintain quality services and create incentives for efficiency~\cite{tarifas},
what makes the electricity trading a strategic instrument in the current power sector~\cite{modeloBR}.

However, the electricity sector has been considered the largest and most complex machine ever built~\cite{ClarkWGellings2009-SmartGrid-char1}.
%Conheço essa frase, aheuahueahea :D
Even the Brazilian customer-centric directives use to be restricted by the dualism of bulk power supply-demand with small granularity.
% economy of scale! e os 3 pilares do setor
Although
% conduct by customer-centric policies and being
this methodology has been
enough to keep the ongoing improvements in this market over the last years,
the prevalent model still limits all kinds of transactions in the whole electricity segment chain,
% The present trading environment
which varies accordingly with consumer power consumption capacity levels, leaving low power consumption customers away from power transactions.

Anyhow, the Brazilian \gls{gd} legislation looks to be ahead of its time and linked to the blockchain business model purpose.
Similarly to how Bitcoin has been allowing profitable trade between small service providers and customers, reducing the infrastructure cost to do so,
the shareable generation model~\cite{GD} is welcoming for such trustless, distributed network of renewable energy sharing and social integration.

% 3. What is the purpose of the present work?
In this way, a blockchain application can ensure the active participation of customers on the power grid for a reliable buy/sell/donate transaction between members of a community, condominium, neighbourhood, municipality, or even different countries~\cite{Coelho2016}.
Following this reasoning, this work has the ambition to contribute with the development of a simple \glsfull{te} blockchain application to be implemented as part of a management platform by Brazilian enterprises of shared generation.

% 4. Present the structure of the document.
Finally, after the contextualization about the reason and goal of the current work,
the following three chapters deep into the above discussion,
while conclusions and suggestions for future improvements are dealt at \autoref{ch:final}.
\autoref{ch:1} seeks to present the micro/mini-grid Brazilian scenario current state, with a systematic view of its operations, and opportunities towards a transactive energy context.
\autoref{appen:1} complements the discussion about the Brazilian commercialization of electric power contextualizing the \gls{gd} approach in the whole sector chain.
\autoref{ch:2} introduces concepts related to the emerging blockchain technology, fundamental for the development of \glspl{dapp}.
Principles about the types of blockchains available are discussed too.
In addition, some examples of blockchain applications on the power sector chain are shown.
\autoref{ch:3} states the proposal itself, with an analysis of the project requirements and the features of the chosen blockchain.
In order to validate the proposal an experiment is conducted.