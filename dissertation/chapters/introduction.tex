Nowadays, modelling intelligent complex systems uses two main paradigms, commonly referred to as Symbolism and Connectionism, as basic guidelines for achieving your goals of creating intelligent machines and understanding human cognition. 
These two approaches depart from different positions, each advocating advantages over the other in reproducing intelligent activity. 
The traditional symbolic approach argues that the algorithmic manipulation of symbolic systems is an appropriate context for modelling cognitive processes. 
On the other hand, connectionists restrict themselves to brain-inspired architectures and argue that this approach has the potential to overcome the rigidity of symbolic systems by more accurately modeling cognitive tasks that can only be solved, in the best case, approximately. 
Years of experimentation with both paradigms lead us to the conclusion that the solution lies between these two extremes, and that the approaches must be integrated and unified. 
In order to establish a proper link between them, much remains to be researched.

If in the 1980s the discussion of intelligence was placed at the distinct poles
of the symbolists and connectionists, today the connectionists are divided by the
reductionist arguments of the structuralists. 
For this structuralist current, the failure of the symbolists was due to the fact that their models despised brain architecture,
and therefore connectionism must continue to explore more deeply the structural aspects
of the thinking organ. 
In this project, the connectionist and structuralist aspects are approached, respectively, through the paradigm of artificial neural networks and realistic models of the brain, within the area called Computational Neuroscience. 
Through our models, we investigate ancient questions of Artificial Intelligence regarding the understanding of computability aspects of the human mind.

In this project we will continue with the study and implementation of Deep Neural Networks (RNPs), which have been used to solve artificial intelligence problems, in areas such as: automatic speech (or voice) recognition, image recognition and treatment, natural language processing, bioinformatics, among many others.

Our previous experience, both in the development of research in the field of artificial neural networks and general distributed processing and its technological applications, as well as in the pursuit of realistic models of brain biology, allows us to mature in the same direction of multidisciplinary research.
