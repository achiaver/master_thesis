Nowadays, modelling intelligent complex systems uses two main paradigms, commonly referred to as Symbolism and Connectionism, as basic guidelines for achieving your goals of creating intelligent machines and understanding human cognition. These two approaches depart from different positions, each advocating advantages over the other in reproducing intelligent activity. The traditional symbolic approach argues that the algorithmic manipulation of symbolic systems is an appropriate context for modelling cognitive processes. On the other hand, connectionists restrict themselves to brain-inspired architectures and argue that this approach has the potential to overcome the rigidity of symbolic systems by more accurately modeling cognitive tasks that can only be solved, in the best case, approximately. Years of experimentation with both paradigms lead us to the conclusion that the solution lies between these two extremes, and that the approaches must be integrated and unified. In order to establish a proper link between them, much remains to be researched.
