``Introdu\c{c}\~{a}o''

Nowadays, modelling intelligent complex systems uses two main paradigms, commonly
referred to as Symbolism and Connectionism, as basic guidelines for achieving your
goals of creating intelligent machines and understanding human cognition. These
two approaches depart from different positions, each advocating advantages over
the other in reproducing intelligent activity. The traditional symbolic approach
argues that the algorithmic manipulation of symbolic systems is an appropriate context
for modelling cognitive processes. On the other hand, connectionists restrict themselves
to brain-inspired architectures and argue that this approach has the potential to
overcome the rigidity of symbolic systems by more accurately modeling cognitive tasks
that can only be solved, in the best case, approximately. Years of experimentation
with both paradigms lead us to the conclusion that the solution lies between these
two extremes, and that the approaches must be integrated and unified. In order to
establish a proper link between them, much remains to be researched.

If in the 1980s the discussion of intelligence was placed at the distinct poles
of the symbolists and connectionists, today the connectionists are divided by the
reductionist arguments of the structuralists. For this structuralist current, the
failure of the symbolists was due to the fact that their models despised brain architecture,
and therefore connectionism must continue to explore more deeply the structural aspects
of the thinking organ. In this project, the connectionist and structuralist aspects
are approached, respectively, through the paradigm of artificial neural networks
and realistic models of the brain, within the area called Computational Neuroscience.
Through our models, we investigate ancient questions of Artificial Intelligence regarding
the understanding of computability aspects of the human mind.

In this project we will continue with the study and implementation of Deep Neural
Networks (RNPs), which have been used to solve artificial intelligence problems,
in areas such as: automatic speech (or voice) recognition, image recognition and
treatment, natural language processing, bioinformatics, among many others.

Our previous experience, both in the development of research in the field of artificial
neural networks and general distributed processing and its technological applications,
as well as in the pursuit of realistic models of brain biology, allows us to mature
in the same direction of multidisciplinary research.


``Justificativas''

In recent years, as RNPs have been used very successfully in various data analysis
tasks. In 2011, for the first time, the use of RNP learning methods enabled the achievement
of the best performance of a human being in a competition to solve visual pattern
recognition problems. These techniques are being used a lot for a solution of various
computational intelligence problems.


``Objetivos''

We will study the problem of pattern recognition in data with Deep Neural Networks.
We will be experimenting with parametric model adjustment on pattern recognition problems
such as image analysis and natural language processing.


``Metas''

\begin{itemize}
  \item Study of the algorithms that make up the Deep Neural Networks technique.
  \item Study of problems that benefit from the use of these RNP techniques.
  \item Experimentation with the parametric adjustment of the model to obtain
  performs well on data pattern recognition issues.
  \item Publication and dissemination of the results of our work.
\end{itemize}


``M\'{e}todo''

We will use Neuronal Networks with deep learning techniques to perform data analysis.
RNPs are a class of machine learning algorithms that use a multilayer cascade with
nonlinear processing units for feature extraction into a dataset. Each successive
layer receives the output signal from the previous layer as input signal. These networks
may use supervised or unsupervised learning techniques. They can learn multiple
levels of representations that correspond to different levels of abstraction.


``Resultado Esperado''

We hope to obtain results that represent advances in understanding the Deep Neural
Network method and how to perform parametric model adjustments to automatically
recognize patterns in problems such as image processing and natural language processing.
