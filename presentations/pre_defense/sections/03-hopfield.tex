\section{Redes de Hopfield}

\begin{frame}{Redes de Hopfield}%
  \justifying%
  Uma rede de Hopfield é uma rede que armazena padrões de tal forma que quando um padrão novo é mostrado para a rede, ela responde devolvendo o padrão que ela tem armazenado mais próximo ao novo padrão.

  Hopfield é uma rede determinística.
\end{frame}

\subsection{Definições}
\begin{frame}{Quais os elementos básicos para Hopfield?}%
  \justifying%
  Vamos considerar que cada uma das \textbf{unidades} da rede é denominada por $\mathrm{x}_{i}$, onde $i = 1, \dots, N$, para uma rede com $N$ unidades.

  Em Hopfield, cada $\mathrm{x}_{i}$ pode assumir um dos valores $x_{i} \in \{0, 1\}$. Rede binária!

  Se $x_{i} = 0$, a unidade $i$ desativada; se $x_{i} = 1$, unidade $i$ ativada.
\end{frame}

\begin{frame}{Quais os elementos básicos para Hopfield?}%
  \justifying%
  Os \textbf{pesos} chamaremos de $\omega$.

  Na rede de Hopfield os pesos são simétricos, isto é, $\omega_{ij} = \omega_{ji}$ (conexão entre as unidades $i$ e $j$).

\end{frame}

\subsection{Esquema}
\begin{frame}{Rede de Hopfield diagrama}
  \begin{figure}[h]{}%
    \label{fig:hopfield}%
    \includegraphics[scale=0.5]{images/hopfield.png}
    \caption{Distribuição dos neurônios de uma rede de Hopfield com 4 unidades.}
  \end{figure}
\end{frame}

\begin{frame}{Rede de Hopfield diagrama}
  \begin{figure}[h]{}%
    \label{fig:hopfield-full}%
    \includegraphics[scale=0.5]{images/hopfield_full.png}
    \caption{Identificação das conexões de uma rede de Hopfield com 4 unidades.}
  \end{figure}
\end{frame}

\begin{frame}{Estado da Rede}%
  \justifying%
  \onslide<1->{%
  Exemplo de um estado $\mathrm{\mathbf{x}}$ para uma rede com 4 unidades:}
  \onslide<2->{%
  \begin{equation}%
    \label{eq:net-state}%
    \mathbf{x} = (0, 1, 1, 0)
  \end{equation}
  }
\end{frame}

\subsection{Funcionamento}
\begin{frame}{Função de Energia}%
  \justifying%
  Segundo \textbf{HOPFIELD (1982)}, como as conexões são simétricas, existe uma função de energia $H$ para a rede, dado o estado $\mathrm{\mathbf{x}}$ em que ela se encontra.
  \begin{equation}%
    \label{eq:hopfield-energy-function}%
    H(\mathbf{x}) = - \frac{1}{2} \sum_{i} \sum_{j} \omega_{ij} x_{i} x_{j} - \sum_{i} \phi_{i} x_{i},
  \end{equation}
  
  Equação~(\ref{eq:hopfield-energy-function}) depende da correlação entre as unidades (primeiro termo), e depende do limiar de ativação de cada unidade (segundo termo). Por simplificação podemos considerar apenas o primeiro termo.

  A energia sempre decresce ou permanece constante conforme o sistema evolui.
  Sistema é estacionário quando a energia é menor. Achou um mínimo.
\end{frame}

\begin{frame}{Aprendizagem}%
  \justifying%
  Fazer a rede de Hopfield aprender corresponde a determinar os valores das conexões $\omega_{ij}$, a partir de um conjuto de padrões apresentado para a rede. Isto é, aprender quais são os mínimos da rede para o conjunto de treinamento em questão.
  \begin{equation}%
    \label{eq:hop-omega}
    \omega_{ij} = \frac{1}{N} x_{i} x_{j},
  \end{equation}
  considerando que a rede aprende apenas um padrão.
\end{frame}

\begin{frame}{Função de Ativação}%
  \justifying%
 O valor $x_{i}$ que a unidade $\mathrm{x}_{i}$ possui é dada pela função degrau.
  \begin{figure}[h]{}%
    \label{fig:hopfield-step}%
    \includegraphics[scale=0.35]{images/hopfield_activation.png}
    \caption{Função de ativação de uma unidade na rede de Hopfield.}
  \end{figure}
  \begin{equation}%
    \label{eq:step-function}
    x_{i} = \Theta \left(\sum_{j} \omega_{ij} x_{j} \right)
  \end{equation}
\end{frame}

\begin{frame}{Procedimento}%
  \justifying%
  Uma vez que a rede determinou o valor de suas conexões, podemos mostrar um novo estado para a rede, por exemplo, $\mathrm{\mathbf{x}} = (1, 1, 0, 0)$.
  
  Partindo de uma unidade escolhida aleatoriamente, determinamos se a unidade fica ativa ou não (muda de valor) avaliando a diferença de energia 
  \begin{equation}%
    \label{eq:energy-delta}%
    \Delta H_{i} = H_{i}(x_{i} = 0) - H_{i}(x_{i} = 1). 
  \end{equation}
  Hopfield tem característica determinista porque a rede sempre opera indo em direção a menor energia. Pode acontecer de ficar preso em mínimo local. Para um problema com muitas variáveis, isso pode ser tornar um problema, pois podemos não chegar ao melhor resultado. Há outras formas de se contornar isso.
\end{frame}
