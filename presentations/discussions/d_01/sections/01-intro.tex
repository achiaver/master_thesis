\section{Introdução}%

\begin{frame}{Introdução}%
  \justifying%
  Redes neuronais artificiais são algoritmos computacionais que se inspiram na rede neuronal biológica.
  Estes algoritmos podem ser usados para resolver problemas onde uma solução analítica é difícil de ser encontrada, por exemplo, no caso de reconhecimento de padrão em imagens.
  \\~\\
  A inspiração nas redes neuronais biologicas, é devido ao fato de que as redes neuronais artificiais tem dois elementos basicos: os neuronios (unidades) e as sinapses (conexoes).
\end{frame}

\begin{frame}{Introdução}%
  \justifying%
  Há diferentes algoritmos de redes neuronais artificiais, suas diferenças estão relacionadas a forma como esses dois elementos básicos são agrupados (como são feitas as conexões entre eles), e como cada unidade opera.
  \\~\\
  Exemplos de redes neuronais conhecidas são: perceptron, MLP, CNN, redes de Hopfield, máquina de Boltzmann, entre outras $\dots$
\end{frame}

\begin{frame}{Motivação e Objetivo}%
  \justifying%
  As diferentes redes neuronais possuem cada uma as suas peculiaridades, e problemas para os quais possuem melhor desempenho se comparadas com seus outros parentes.
  \\~\\
  Neste trabalho o nosso objetivo é entender o lado teórico da máquina de Boltzmann, e de suas derivadas, como a máquina restrita de Boltzmann, e aplicá-las em um problema simples afim de comprovar seu mecanismo de funcionamento.
\end{frame}
